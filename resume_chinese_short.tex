% !TEX program = xelatex

\documentclass{resume}

\usepackage{hyperref}
\usepackage[UTF8]{ctex}

\begin{document}

\pagenumbering{gobble} % suppress displaying page number

\name{叶耀阳}

\basicInfo{
  \email{y@cubicy.icu} \textperiodcentered\
  \phone{(+86) 182-5882-5609} \textperiodcentered\
  \github[CubicYYY]{https://github.com/CubicYYY}
  \homepage[Blog: cubicy.icu]{https://cubicy.icu}}

\section{\faGraduationCap\ 教育经历}

\datedsubsection{\textbf{浙江大学(Zhejiang University)}}{2021.9 - 2025.6(预计)}
\textit{本科, 计算机科学与技术}
\begin{itemize}[parsep=0.5ex]
  \item GPA: 3.91/4.0
  \item Junior year GPA: \textbf{3.95/4.0 (91.37/100)}
  \item 主要课程:高级数据结构与算法分析、计算机组成、博弈论、概率论与数理统计、编程语言原理
  \item 校级荣誉:浙江大学二等奖学金(2022, 2023),浙江大学三等奖学金(2021)
\end{itemize}

\datedsubsection{\textbf{东京大学(The University of Tokyo)}}{2024.9 - 2025.3}
\textit{校级交换项目(USTEP)}\\

\section{\faCogs\ 工作经历}

\datedsubsection{\textbf{Humanify AI}}{2025.1 至今}
\role{远程实习}{系统开发}
\begin{itemize}[parsep=0.5ex]
  \item 基于 AOSP 定制 AI 原生系统,服务于用户硬件终端上的模型推理
  \item 基于 ZeroMQ 设计低延迟、少拷贝的进程间通信架构
  \item 使用 Rust 实现内核驱动等高可靠系统组件
\end{itemize}

\datedsubsection{\textbf{上海洛谷网络科技有限公司}}{2021 - 2022}
\role{远程实习}{后端开发/讲师}
\begin{itemize}[parsep=0.5ex]
  \item 通过修改弹性扩容策略,优化动态负载下的在线代码运行性能
  \item 担任部分算法竞赛课程主讲
\end{itemize}

\datedsubsection{\textbf{浙江大学启真交叉学科创新创业实验室(X-Lab)}}{2022.7 至今}
\role{成员}{全栈开发}
\begin{itemize}[parsep=0.5ex]
  \item 与浙江大学历史学院合作,作为技术团队负责人,开发清代科举人物数据库与配套网站
  \item 不定期进行技术分享、教学
\end{itemize}

\datedsubsection{\textbf{浙江大学 Azure Assassin Alliance (AAA) 战队}}{2021 至今}
\role{成员}{Web / Misc}
\begin{itemize}[parsep=0.5ex]
  \item 作为成员,参与国内外夺旗赛(CTF),获得 SECCON 2023 团队亚军(全球)等荣誉
  \item 作为讲师,负责浙江大学暑学期网络攻防实践课程 Web 方向部分内容
\end{itemize}

\section{\faCogs\ 项目}

\datedsubsection{\textbf{与 Jingbo Wang 教授(Purdue 大学)的合作研究}}{2023.11 至今}
\role{学术}{远程研究助手}
\begin{itemize}[parsep=0.5ex]
  \item 通过 LLM 组件加速程序综合(Program Synthesis)
  \item 使用 XGrammar 组件加速语法树生成、搜索,更早得到所需代码片段
\end{itemize}

\datedsubsection{\textbf{策略生成与回测框架}}{2022.7 至今}
\role{个人项目}{Python}
\begin{itemize}[parsep=0.5ex]
  \item 通过领域特定语言(DSL)定义策略,允许对带参策略的参数取值进行搜索、调优
  \item 目前尝试使用 XGrammar 根据 DSL 语法自动生成策略
  \item 近期目标:实现策略生成-调优-回测的全自动闭环
\end{itemize}

\section{\faCogs\ 技能}
\begin{itemize}[parsep=0.5ex]
  \item 编程技能:C++, Rust, Python, JavaScript, PHP, MATLAB, Shell, SQL, Haskell
  \item 语言能力:英语 - TOEFL 100, CET6 598; 日语 - 中等(日常交流)
\end{itemize}

\section{\faInfo\ 其他}
\begin{description}[parsep=0.5ex]
  \item[个人博客] http://cubicy.icu
  \item[GitHub] https://github.com/CubicYYY
  \item[升学] 已获得 Purdue University 的 PhD offer(人工智能,编程语言理论和信息安全交叉方向)
  \item[论文] 目前尚有一篇论文在投
\end{description}

\end{document}

